\section{Einleitung}

Munin ist ein Monitoring-Tool, welches Computer, Services und Ressourcen überwacht und die entsprechenden Daten abspeichert. Alle Informationen werden in graphischen Schaubildern durch ein Webinterface dargestellt.

Als Überwachungsmechanismus setzt Munin auf Schwellwerte, die durch den Benutzer frei anpassbar sind.
Werden diese Werte über- bzw. unterschritten, gibt Munin eine Warnung aus oder schlägt, bei einer kritischen Überschreitung, entsprechend Alarm.

Im Gegensatz zu den meisten Konkurrenzprodukten legten die Entwickler von Munin den Schwerpunkt auf einen simplen Aufbau bzw. eine schnelle und unkomplizierte Inbetriebnahme der Anwendung.
Bereits nach der Installation stehen eine hohe Anzahl von Plugins zur Verfügung.
Doch auch eigene Munin-Skripte lassen sich, aufgrund des simplen Aufbaus, schnell und unkompliziert selbst programmieren.

Das Munin mehr Wert auf die Visualisierung der Überwachungsdaten legt, als auf eine komplexe Überwachung- oder Alarmierungslogik, stellt sich im direkten Vergleich mit Nagios oder bereits durch das schlicht gehaltene Webinterface heraus.
Dafür lässt sich Munin recht einfach in ein bereits vorhandenes Nagios-Überwachungssystem integrieren und somit können die Vorteile beider Anwendungen genutzen werden.