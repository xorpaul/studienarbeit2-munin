\section{Zusammenfassung und Ausblick}
Die von Munin zusammengetragenen Informationen sind dank der einfachen Installation der Anwendung und der gewünschten Munin-Plugins sehr schnell in nützliche Graphen umgewandelt und erweisen sich dank der unterschiedlichen Zeitauflösung als sehr nützliches Element in der Langzeitüberwachung.

Auch die große Anzahl an bereits bei der Installation verfügbaren Plugins können, dank einer automatischen Testfunktion 
%, siehe Abbildung \ref{suggest} in Kapitel \ref{umsetzung}, 
schnell und unkompliziert eingebunden werden.

Sollte sich doch nicht das passende Plugin in dieser Sammlung finden, gibt es die Möglichkeit auf der MuninExchange Internetseite fündig zu werden.
Dort befinden sich von anderen Munin-Benutzern entwickelte und veröffentlichte Plugins, die auch weiter vom Benutzer auf die eigenen Bedürfnisse angepasst werden können.
Da die Plugins jedoch recht einfach aufgebaut sind, lassen sich auch - schnell und einfach - eigene Überwachungsskripte entwickeln, die sofort automatisch nach der Verlinkung zum Service-Verzeichnis in das Webinterface eingefügt werden.

Munin legt großen Wert auf die Visualisierung der Überwachungsinformationen und weniger auf eine komplexe, umfangreiche Überwachungs- und Alarmierungslogik.
Für diesen Zweck empfiehlt es sich Nagios zu verwenden und Munin in das Nagios Überwachungssytem einzubinden.
Denn sowohl Nagios als auch Munin bieten die notwendigen Werkzeuge um dies zu realisieren.
Dabei muss erwähnt werden, dass auch mit Nagios die Visualisierung der Performancedaten der Plugins möglich ist, so dass Munin als überflüssig gesehn werden könnte.
Jedoch stellt sich die Konfiguration und Anpassung von Nagios und dem Visualisierung-Addon für den unerfahrenen Benutzer als deutlich schwieriger und aufwändiger heraus, als der Aufbau eines ähnlichen Munin Überwachungssystems.